Lembaga Penelitian dan Pengabdian kepada
Masyarakat(LPPM) merupakan salah satu lembaga yang ada di setiap Universitas di Indonesia.
Lembaga ini bertugas mengelola penelitian dan pengabdian kepada masyarakat yang dilakukan
oleh dosen yang ada di Universitas. Melaksanakan penelitian dan pengabdian kepada masyarakat
merupakan bagian dari Tridharma Perguruan Tinggi yang harus dilaksanakan oleh setiap dosen dan sivitas akademika.
Dalam pengelolaan kegiatan tersebut, dibutuhkan sebuah sistem informasi yang
dapat membantu pengelolaan data kegiatan, sehingga penngelolaan data kegiatan dapat
dilakukan dengan lebih efektif, efisien, dan terintegrasi.
Metodologi yang digunakan dalam penelitian ini adalah
metode \textit{Software Development Life Cycle \emph{(SDLC)}} waterfall, dimana metode ini 
terdiri dari beberapa tahapan yang berurutan mulai dari analisis kebutuhan hingga implementasi sistem.
Sistem informasi ini akan dibangun menggunakan MERN \textit{Stack Development} yang tersusun 
dari MongoDB sebagai \textit{database}, Express sebagai \textit{back-end}, React sebagai 
\textit{front-end}, dan Node.js sebagai \textit{runtime environment}.
Hasil penelitian ini yaitu Sistem Informasi berbasis web
untuk mengelola data penelitian dan PkM dengan berbagai fitur yang 
memungkinkan pengelola LPPM maupun dosen melakukan pengusulan, 
pencarian kembali, dan pengunggahan dokumen yang terkait dengan 
kegiatan penelitian dan/atau PkM.
Sistem Informasi yang dikembangkan dapat membantu LPPM dalam
pengelolaan data kegiatan penelitian dan PkM, sehingga pengelolaan data dapat
dilakukan dengan lebih efektif, efisien, dan terintegrasi.

\noindent{Kata kunci} : Sistem Informasi, LPPM, \textit{Waterfall}, MERN \textit{Stack Development}

\vspace{1cm}

%HAPUS YANG TIDAK PERLU
%-------------------------------------------------
% \noindent\fbox{%
% 	\parbox{\textwidth}{%
% \textbf{Contoh Abstrak Teknik Elektro:} \\

% \hspace{1cm} "Penelitian ini bertujuan untuk mengembangkan sistem pengendalian suhu ruangan dengan menggunakan microcontroller. Metodologi yang digunakan adalah desain sirkuit, implementasi sistem pengendalian, dan pengujian performa. Hasil penelitian menunjukkan 
% bahwa sistem pengendalian suhu ruangan yang dikembangkan mampu mengendalikan suhu ruangan dengan akurasi sebesar ±0,5°C. Kesimpulan dari penelitian ini adalah sistem pengendalian suhu ruangan yang dikembangkan efektif dan efisien. \\

% Kata kunci: microcontroller, sistem pengendalian suhu, akurasi."
% \vspace{5mm}

% \textbf{Contoh Abstrak Teknik Biomedis:} \\

% \hspace{1cm} "Penelitian ini bertujuan untuk mengevaluasi keefektifan prototipe alat pemantau denyut jantung berbasis elektrokardiogram (ECG) untuk pasien jantung. Metodologi yang digunakan meliputi desain dan pembuatan prototipe, pengujian dengan pasien, dan analisis data. Hasil penelitian menunjukkan bahwa prototipe alat pemantau denyut jantung berbasis ECG memiliki 
% akurasi yang baik dan mampu memantau denyut jantung pasien secara efektif. Kesimpulan dari penelitian ini adalah prototipe alat pemantau denyut jantung berbasis ECG merupakan solusi 
% yang efektif dan efisien untuk memantau pasien jantung. \\

% Kata kunci: elektrokardiogram, alat pemantau denyut jantung, akurasi."
% \vspace{5mm}

% 	}%
% }

% %-------------------------------------------------

% \noindent\fbox{%
% 	\parbox{\textwidth}{%
% 		\textbf{Contoh Abstrak Teknologi Informasi:} \\
		
% \hspace{1cm} "Penelitian ini bertujuan untuk mengevaluasi keamanan dan privasi pengguna aplikasi media sosial terpopuler. Metodologi yang digunakan meliputi analisis kebijakan privasi dan pengaturan keamanan, pengujian penetrasi, dan survei pengguna. Hasil penelitian 
% menunjukkan bahwa beberapa aplikasi media sosial memiliki kebijakan privasi yang kurang jelas dan rendahnya tingkat keamanan. Kesimpulan dari penelitian ini adalah pentingnya meningkatkan kebijakan privasi dan tingkat keamanan pada aplikasi media sosial untuk melindungi privasi dan data pengguna. \\
		
% Kata kunci: media sosial, keamanan, privasi, pengguna."
% \vspace{5mm}
		
% 	}%
% }