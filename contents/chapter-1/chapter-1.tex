\chapter{Pendahuluan}

\section{Latar Belakang}

Setiap perguruan tinggi di Indonesia wajib menyelenggarakan kegiatan penelitian dan 
pengabdian kepada masyarakat sebagaimana yang tercantum dalam Undang-Undang Republik
Indonesia Nomor 12 Tahun 2012 tentang Pendidikan Tinggi. Dalam Undang-Undang tersebut
juga disebutkan bahwa kegiatan penelitian dan pengabdian kepada masyarakat merupakan
dua dari tiga unsur Tridharma Perguruan Tinggi yang merupakan amanah yang harus 
dilaksanakan oleh setiap perguruan tinggi di Indonesia. Dalam hal ini, penelitian yang
dimaksud yaitu kegiatan berdasarkan kaidah dan metode ilmiah yang sistematis dan bertujuan 
untuk memperoleh informasi atau data berdasarkan suatu cabang ilmu pengetauan. Sedangkan
pengabdian kepada masyarakat merupakan kegiatan yang bertujuan untuk memajukan kesejahteraan 
masyarakat dengan memanfaatkan pengetahuan dan teknologi yang telah ada.

Dalam melaksanakan kegiatan penelitian dan pengabdian kepada masyarakat, setiap perguruan
tinggi di Indonesia wajib memiliki lembaga yang bertugas mengelola kegiatan dan dokumen yang
terkait dengan kegiatan tersebut. Lembaga tersebut dikenal dengan nama Lembaga Penelitian
dan Pengabdian kepada Masyarakat (LPPM). LPPM merupakan lembaga yang bertanggung jawab
langsung kepada rektor dalam mengelola kegiatan penelitian dan pengabdian kepada masyarakat
di lingkungan perguruan tinggi. LPPM juga bertugas untuk mengelola dokumen yang terkait dengan
kegiatan penelitian dan pengabdian kepada masyarakat seperti proposal penelitian dan pengabdian
kepada masyarakat, laporan penelitian dan pengabdian kepada masyarakat, dan dokumen lainnya.

Pada zaman yang serba digital ini, sudah semestinya pengelolaan kegiatan penelitian dan 
pengabdian kepada masyarakat dilakukan menggunakan solusi digital. Solusi digital yang 
dimaksud adalah dengan menggunakan sistem informasi. Sistem informasi ini dapat membantu
LPPM dalam mengelola dokumen yang terkait penelitian dan pengabdian kepada
masyarakat seperti proposal, laporan akhir, dan dokumen lainnya. Sistem informasi ini juga 
dapat memudahkan LPPM dalam mengelola jalannya kegiatan penelitian dan pengabdian kepada 
masyarakat seperti pengajuan proposal , pemilihan \textit{reviewer} proposal, proses 
\textit{review} proposal, \textit{monitoring} pelaksanaan, serta pengumpulan laporan
akhir penelitian dan pengabdian kepada masyarakat. Selain itu, sistem informasi ini juga
dapat memudahkan LPPM dalam mem-\textit{posting} pengumuman terkait pembukaan proposal penelitian
dan pengabdian kepada masyarakat yang nantinya juga akan memudahkan dosen dalam mengetahui
pengumuman tersebut.

% Sub-bab ini berisi uraian tentang latar belakang atau justifikasi ilmiah dan permasalahan yang akan diteliti, alasan penelitian dan penelitian yang pernah dilakukan sebelumnya terkait fenomena tersebut.

%HAPUS YANG TIDAK PERLU
%-------------------------------------------------
% \noindent\textbf{Contoh latar belakang penelitian untuk teknologi informasi:} \\
% \noindent\fbox{%
% 	\parbox{\textwidth}{%
		
% \hspace{1cm} "Dengan perkembangan teknologi informasi yang sangat pesat dalam beberapa tahun terakhir, penyimpanan data menjadi masalah yang semakin penting. Semakin banyak data yang diterima setiap hari, semakin penting bagi organisasi untuk memastikan bahwa data 
% mereka aman dan terlindungi. Pada saat yang sama, organisasi juga membutuhkan akses cepat dan efisien ke data mereka untuk membuat keputusan yang tepat. Teknologi 
% enkripsi kuantum baru-baru ini muncul sebagai solusi potensial untuk memenuhi kebutuhan ini, dengan menawarkan tingkat keamanan yang jauh lebih tinggi dan proses 
% enkripsi yang lebih cepat dibandingkan dengan teknologi enkripsi konvensional. Oleh karena itu, penting untuk mengevaluasi efektivitas dan keamanan teknologi enkripsi 
% kuantum dalam sistem penyimpanan data cloud."
		
% 	}%
% }

% %-------------------------------------------------	
% \vspace{5mm}
% Latar belakang ini memperkenalkan masalah penyimpanan data dan mempertimbangkan pentingnya keamanan data. Ini juga memperkenalkan teknologi enkripsi kuantum sebagai solusi potensial dan menjelaskan mengapa evaluasi teknologi ini penting bagi bidang teknologi informasi. Latar belakang ini memberikan dasar yang kuat bagi perumusan masalah dan tujuan penelitian, memastikan bahwa hasil penelitian memiliki relevansi dan signifikansi bagi bidang terkait.

\section{Rumusan Masalah}

Berdasarkan latar belakang yang telah disebutkan pada bagian sebelumnya, 
rumusan masalah penelitian ini adalah bagaimana mengembangkan sistem informasi
pengelolaan data dan pengabdian kepada masyarakat?

% Sub-bab ini berisi poin-poin yang menjelaskan masalah atau isu yang akan diteliti dalam suatu penelitian. Ini mencakup formulasi masalah yang jelas dan spesifik dan 
% mempertimbangkan latar belakang dan deskripsi masalah yang terkait yang tertulis Dalam latar belakang pada sub-bab sebelumnya. Perumusan masalah yang baik akan membantu menentukan fokus penelitian dan memberikan dasar untuk tujuan dan hipotesis penelitian. Dalam penelitian, perumusan masalah memainkan peran penting dalam memastikan bahwa hasil penelitian berguna dan relevan untuk masalah yang diteliti.
% %-------------------------------------------------	
% \noindent\fbox{%
% 	\parbox{\textwidth}{%
% \noindent\textbf{Contoh} perumusan masalah untuk \textbf{Teknologi Informasi}: \\		

% \hspace{1cm} \textbf{"Bagaimana meningkatkan efisiensi dan keamanan sistem penyimpanan data cloud melalui implementasi teknologi enkripsi kuantum?"} \\

% \hspace{1cm} Perumusan masalah ini jelas dan spesifik dan menentukan fokus penelitian pada peningkatan efisiensi dan keamanan sistem penyimpanan data cloud dengan menggunakan teknologi enkripsi kuantum. Ini mempertimbangkan latar belakang tentang pentingnya keamanan data dan memberikan solusi praktis melalui implementasi teknologi. Perumusan masalah ini memberikan dasar yang kuat untuk tujuan dan hipotesis penelitian dan memastikan bahwa hasil penelitian akan berguna bagi bidang teknologi informasi.
		
% 	}%
% }

%-------------------------------------------------	

\section{Tujuan Penelitian}

Tujuan penelitian ini adalah membangun sistem informasi pengelolaan data penelitian dan pengabdian 
kepada masyarakat yang memungkinkan pengelola LPPM dan dosen melakukan pengusulan, pencarian 
kembali, dan pengunggahan dokumen yang terkait dengan kegiatan penelitian dan/atau pengabdian
kepada masyarakat.
% Tujuan penelitian pada skripsi Teknik (TE, TB, TIF) adalah menentukan sasaran atau target yang ingin dicapai melalui proses penelitian. Tujuan penelitian bisa beragam sesuai dengan bidang ilmu yang dipelajari, topik penelitian, dan permasalahan yang akan dicari solusinya.

% \begin{itemize}
% 	\item Men-\textit{develop} sistem informasi untuk pengelolaan data penelitian dan pengabdian kepada masyarakat.
% 	\item Meningkatkan pemahaman dan wawasan tentang bidang \textit{engineering} melalui penerapan teori dan metodologi yang sesuai.
% 	\item Mengembangkan solusi atau produk baru yang inovatif dan efektif dalam 
% 	bidang \textit{engineering}.
% 	\item Menunjukkan penerapan prinsip-prinsip keteknikan dalam solusi atau produk yang dikembangkan.
% 	\item Menjelaskan implikasi dan rekomendasi dari hasil penelitian bagi bidang \textit{engineering} dan masyarakat.
% \end{itemize}
% \begin{itemize}
% 	\item Mengidentifikasi dan menganalisis masalah atau permasalahan dalam bidang \textit{engineering}.
% 	\item Meningkatkan pemahaman dan wawasan tentang bidang \textit{engineering} melalui penerapan teori dan metodologi yang sesuai.
% 	\item Mengembangkan solusi atau produk baru yang inovatif dan efektif dalam 
% 	bidang \textit{engineering}.
% 	\item Menunjukkan penerapan prinsip-prinsip keteknikan dalam solusi atau produk yang dikembangkan.
% 	\item Menjelaskan implikasi dan rekomendasi dari hasil penelitian bagi bidang \textit{engineering} dan masyarakat.
% \end{itemize}

% \begin{minipage}{0.92\textwidth}
% Catatan: Tujuan penelitian dalam skripsi bidang teknik harus jelas, spesifik, terukur, dan dapat dicapai dalam jangka waktu yang ditentukan.
% \end{minipage}

% \newpage
% \vspace{5mm}
% \textbf{Contoh Tujuan Penelitian Skripsi Teknik Elektro:}

% \begin{minipage}{0.92\textwidth}
% Berikut adalah beberapa contoh tujuan penelitian yang sesuai dengan topik 
% “perbaikan efisiensi penghematan energi pada sistem pencahayaan rumah tangga 
% melalui implementasi teknologi kontrol otomatis”:
% \end{minipage}

% %-------------------------------------------------	
% \noindent\fbox{%
% 	\parbox{\textwidth}{%
% \begin{enumerate}
% \item Menganalisis tingkat efisiensi energi pada sistem pencahayaan rumah tangga 
% sebelum dan setelah implementasi teknologi kontrol otomatis.
% \item Mengukur pengurangan biaya listrik setelah implementasi teknologi kontrol 
% otomatis pada sistem pencahayaan rumah tangga.
% \item Menunjukkan bagaimana teknologi kontrol otomatis dapat memperbaiki 
% efisiensi penghematan energi pada sistem pencahayaan rumah tangga.
% \item Meningkatkan kenyamanan dan keamanan pengguna rumah tangga melalui 
% penggunaan teknologi kontrol otomatis pada sistem pencahayaan.
% \item Menjelaskan bagaimana implementasi teknologi kontrol otomatis 
% mempengaruhi kualitas cahaya dan faktor-faktor lain dalam sistem pencahayaan 
% rumah tangga.
% \item Membandingkan efisiensi energi dan biaya pada sistem pencahayaan rumah 
% tangga dengan teknologi kontrol otomatis dan sistem manual.
% \item Menunjukkan implikasi dan rekomendasi dari hasil penelitian bagi rumah tangga 
% dan lingkungan.
% \end{enumerate}
		
% 	}%
% }

% %-------------------------------------------------	

% \newpage
% \vspace{5mm}
% \textbf{Contoh Tujuan Penelitian Skripsi Teknik Biomedis:}

% \begin{minipage}{0.92\textwidth}
% Berikut adalah beberapa contoh tujuan penelitian untuk penelitian dengan tema "Bagaimana memperbaiki akurasi deteksi kanker payudara dengan menggunakan 
% teknologi pemindaian ultrasonografi berbasis AI":
% \end{minipage}

% %-------------------------------------------------	
% \noindent\fbox{%
% 	\parbox{\textwidth}{%
% \begin{enumerate}
% \item Mengidentifikasi faktor-faktor yang mempengaruhi akurasi deteksi kanker 
% payudara dengan menggunakan teknologi pemindaian ultrasonografi berbasis 
% AI.
% \item Menilai efektivitas teknologi pemindaian ultrasonografi berbasis AI dalam 
% meningkatkan akurasi deteksi kanker payudara.
% \item Menentukan metode pemindaian ultrasonografi berbasis AI yang paling efektif dalam meningkatkan akurasi deteksi kanker payudara.
% \item Menilai keamanan dan tolerabilitas teknologi pemindaian ultrasonografi berbasis AI dalam deteksi kanker payudara.
% \item Membandingkan akurasi deteksi kanker payudara dengan teknologi pemindaian 
% ultrasonografi berbasis AI dengan metode deteksi lainnya.
% \item Menyediakan bukti ilmiah untuk menunjukkan bahwa teknologi pemindaian 
% ultrasonografi berbasis AI dapat digunakan sebagai metode deteksi kanker 
% payudara yang lebih efektif dan akurat.
% \item Meningkatkan akurasi deteksi kanker payudara dengan menggunakan teknologi 
% pemindaian ultrasonografi berbasis AI.
% \end{enumerate}
		
% 	}%
% }

% %-------------------------------------------------	

% \newpage
% \vspace{5mm}
% \textbf{Contoh Tujuan Penelitian Skripsi Teknologi Informasi:}

% \begin{minipage}{0.92\textwidth}
% Berikut adalah beberapa contoh tujuan penelitian untuk penelitian dengan tema 
% "Bagaimana meningkatkan efisiensi dan keamanan sistem penyimpanan data cloud 
% melalui implementasi teknologi enkripsi kuantum?":
% \end{minipage}

% %-------------------------------------------------	
% \noindent\fbox{%
% 	\parbox{\textwidth}{%
% \begin{enumerate}
% \item Menilai efektivitas implementasi teknologi enkripsi kuantum dalam 
% meningkatkan keamanan data pada sistem penyimpanan cloud.
% \item Menentukan metode enkripsi kuantum yang paling efektif dalam meningkatkan 
% keamanan data pada sistem penyimpanan cloud.
% \item Membandingkan efisiensi enkripsi kuantum dengan metode enkripsi lainnya 
% dalam meningkatkan keamanan data pada sistem penyimpanan cloud.
% \item Menilai keamanan dan stabilitas sistem penyimpanan data cloud setelah 
% implementasi teknologi enkripsi kuantum.
% \item Menyediakan bukti ilmiah untuk menunjukkan bahwa implementasi teknologi 
% enkripsi kuantum dapat meningkatkan efisiensi dan keamanan sistem 
% penyimpanan data cloud.
% \item Mengidentifikasi potensi masalah dan hambatan dalam implementasi teknologi 
% enkripsi kuantum pada sistem penyimpanan data cloud.
% \end{enumerate}
		
% 	}%
% }

%-------------------------------------------------	

\section{Batasan Penelitian}

Beberapa batasan penelitian ini adalah sebagai berikut:

\begin{enumerate}
\item Objek penelitian: pengembangan sistem informasi pengelolaan data penelitian dan pengabdian 
kepada masyarakat
\item Metode penelitian: penelitian menggunakan metode \textit{Software Development Life Cycle} (SDLC) dengan
tahapan analisis kebutuhan, perancangan sistem desain, dan implementasi.
\item Waktu dan tempat penelitian: 
	\begin{enumerate}
		\item Waktu penelitian: 6 bulan (Januari - Agustus 2023)
		\item Tempat penelitian: Perpustakaan Fakultas Teknik UGM, DTETI FT UGM, dan rumah pribadi.
	\end{enumerate}
% \item Populasi dan sampel: jelaskan populasi dan sampel yang akan diteliti, misalnya produk, mesin, atau sistem.
% \item Variabel dan hipotesis: jelaskan variabel yang akan diteliti dan hipotesis yang akan dibuktikan atau ditolak.
\item Keterbatasan Penelitian: Keterbatasan penelitian adalah metodologi pengembangan sistem Informasi
yang diterapkan adalah analisis kebutuhan, perancangan sistem desain, dan implementasi.
\end{enumerate}

% \newpage
% \noindent \textbf{Contoh penulisan batasan skripsi Teknologi Informasi:}

% %-------------------------------------------------	
% \noindent\fbox{%
% 	\parbox{\textwidth}{%
% \begin{enumerate}
% \item Objek Penelitian: Analisis perbandingan efektivitas dan efisiensi antara sistem manajemen proyek tradisional dan sistem manajemen proyek berbasis teknologi informasi. 
% \item Metode Penelitian: Penelitian kualitatif dengan menggunakan wawancara dan 
% survei terhadap para pelaku proyek di berbagai perusahaan. 
% \item Waktu dan Tempat Penelitian: Waktu penelitian adalah Januari-Juni 2022 di 
% perusahaan-perusahaan di wilayah Bantul. 
% \item Populasi dan Sampel: Populasi nya adalah perusahaan yang melakukan proyek, dan sampel diambil sebanyak 10 perusahaan yang menerapkan sistem manajemen 
% proyek tradisional dan 10 perusahaan yang menggunakan sistem manajemen 
% proyek berbasis teknologi informasi. 
% \item Variabel: Variabel bebasnya adalah sistem manajemen proyek, dan variabel 
% terikatnya adalah efektivitas dan efisiensi. 
% \item Hipotesis: bahwa sistem manajemen proyek berbasis teknologi informasi lebih efektif dan efisien dibandingkan dengan sistem manajemen proyek tradisional.
% \item Keterbatasan Penelitian: Keterbatasan penelitian adalah penelitian hanya dilakukan pada perusahaan di wilayah Bantul dan hanya melibatkan wawancara dan survei sebagai metode pengumpulan data.
% \end{enumerate}
		
% 	}%
% }

\section{Manfaat Penelitian}

Penelitian berupa pengembangan sistem informasi ini diharapkan dapat bermanfaat:

\begin{enumerate}
	\item untuk mempermudah pengelolaan data penelitian dan pengabdian kepada masyarakat
	bagi LPPM.
	\item sebagai bahan referensi dalam merancang desain sistem sebuah sistem 
	informasi pengelolaan data penelitian dan pengabdian kepada masyarakat.
	\item sebagai bahan pembelajaran bagi mahasiswa yang ingin mengembangkan
	sebuah aplikasi web.
\end{enumerate}


\section{Sistematika Penulisan}

Sistematika penulisan laporan penelitian ini adalah sebagai berikut:
% Sistematika penulisan berisi pembahasan apa yang akan ditulis di setiap bab. 
% Sistematika pada umumnya berupa paragraf yang setiap paragraf mencerminkan 
% bahasan setiap Bab. Contoh:

\noindent Bab I membahas tentang pendahuluan yang berisi latar belakang, perumusan masalah 
dan tujuan penelitian. 

\noindent Bab II berisi tentang tinjauan pustaka berupa penelitian terdahulu yang terkait dengan
penelitian ini dan dasar teori berupa teori-teori yang mendukung penelitian ini.

\noindent Bab III berisi tentang berbagai alat, baik \textit{software} maupun \textit{hardware}, 
dan bahan yang digunakan dalam pengerjaan tugas akhir, metodologi penelitian yang terdiri 
dari tahapan analisis kebutuhan, perancangan sistem desain, implementasi, dan pengujian sistem.

\noindent Bab IV berisi 

