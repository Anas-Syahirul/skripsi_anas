\chapter{Metode Penelitian}

Bab ini menjelaskan metode atau cara yang digunakan dalam penelitian ini untuk 
mencapai maksud dan tujuan seperti yang tertulis dalam sub-bab 1.3 [jika diinginkan, kalian dapat menuliskan Kembali tujuan penelitian yang ingin dicapai di sini].

\section{Alat dan Bahan Tugas akhir}

\subsection{Alat Tugas akhir}

Alat-alat yang digunakan pada tugas akhir ini berupa perangkat keras maupun perangkat lunak dengan spesifikasi sebagai berikut:

\begin{enumerate}
	\item \textit{Notebook} ASUS Vivobook dengan spesifikasi sistem operasi Windows 11, \textit{processor} Intel Core i5-1235U, 16GB RAM DDR4, Intel Iris Xe, SSD 500GB.
	\item \textit{Browser} Microsoft Edge untuk menjalankan aplikasi \textit{front-end}.
	\item \textit{Text Editor} Visual Studio Code untuk \textit{coding} aplikasi \textit{front-end} dan \textit{back-end}.
	\item MongoDB Atlas \textit{dashboard} untuk mengelola \textit{database} sistem informasi.
	\item Cloudinary \textit{Digital Asset Management} untuk mengelola \textit{cloud storage} dengan data yang berbentuk media seperti gambar dan  \textit{document}
	\item Aplikasi diagrams.net untuk merancang \textit{flowchart, Entity Relationship Table, dan Use Case Diagram}.
	\item ReactJs versi 18.2.0 untuk \textit{front-end} aplikasi.
	\item NodeJs versi 18.15.0 untuk \textit{runtime environment} aplikasi \textit{back-end}.
	\item ExpressJs versi 4.18.2 untuk \textit{framework} aplikasi \textit{back-end}.
	\item \textit{Node Package Manager} (NPM) versi 9.5.0 untuk mengelola \textit{dependencies} aplikasi.
	\item Json Web Token versi 9.0.0 untuk proses \textit{authentication} dan \textit{authorization}.
	\item Cloudinary Node SDK versi 1.35.0 untuk menghubungkan aplikasi dengan \textit{cloud storage} Cloudinary.
	\item Mongoose versi 7.0.11 untuk menghubungkan aplikasi dengan \textit{database} MongoDB.
	\item Material UI versi 5.13.0 untuk mengelola komponen \textit{User Interface} pada aplikasi.
	\item Postman untuk melakukan \textit{testing} API.
\end{enumerate}
\subsection{Bahan Tugas akhir}

Bahan yang digunakan pada tugas akhir ini yaitu berupa data yang diperoleh dari Dosen 
Pembimbing yaitu berupa \textit{requirement} sistem informasi yang akan dikembangkan. 
Detail mengenai bagaimana \textit{requirement} tersebut didapatkan akan dijelaskan pada Bab 3.

% Bahan tugas akhir adalah segala sesuatu yang bersifat fisik atau digital yang digunakan untuk kebutuhan tugas akhir. Bahan tugas akhir dapat berupa:

% \begin{enumerate}
% 	\item Bahan habis pakai. Bahan yang digunakan untuk tugas akhir. Sebagai contoh 
% 	mungkin dibutuhkan kertas transparansi, baterai, atau yang lain 
% 	\item Bahan yang berupa data atau informasi yang menjadi dataset tugas akhir. Dataset tugas akhir dapat berupa:
% \end{enumerate}
% \begin{itemize}
% 	\item Dataset pihak lain yang diperoleh dengan izin atau dalam lisensi yang diizinkan untuk digunakan secara langsung 
% 	\item Dataset pihak pertama yang disusun sendiri melalui quisioner, observasi, atau interview 
% 	\item Dokumen panduan yang mengacu pada standar, hasil tugas akhir, atau artikel yang disitasi dan digunakan.
% \end{itemize}


\section{Metode yang Digunakan}

Bagian ini membahas metode atau cara yang akan digunakan dalam penelitian, tahapan 
penerapan metode, dan desain penelitian (misalnya apakah penelitian akan menggunakan 
eksperimen di Laboratorium atau di lapangan, misalkan saja penelitian biomedis atau 
penelitian alat ukur hama yang dapat dilakukan di laboratorium ataupun di lapangan, atau menggunakan metode survei (misalnya untuk teknologi Informasi), studi kasus, atau analisis dengan perangkat lunak (ETAP, LTSpice, dst), atau \textit{prototyping} (pembuatan perangkat keras).

Bagian ini juga membahas bagaimana data [akan] dianalisis, apakah dengan membandingkan keluaran beberapa alat ukur, membandingkan dengan standar atau bagaimana.

\section{Alur Tugas Akhir}

Menguraikan prosedur yang akan digunakan dan jadwal atau alur penyelesaian setiap 
tahap. Alur penelian ini dapat disajikan dalam bentuk diagram. Diagram dapat disusun dengan aturan yang baik semisal menggunakan \textit{flowchart}. Aturan dan tutorial pembuatan \textit{flowchart} dapat dilihat di \textcolor{blue}{http://ugm.id/flowcharttutorial}. Setelah menggambarkannya, penulis wajib menjelaskan langkah-langkah setiap alur tugas akhir dalam sub bab tersendiri sesuai dengan kebutuhan.

% \section{Etika, Masalah, dan Keterbatasan Penelitian (Opsional))}

% Bagian ini membahas pertimbangan etis penelitian dan [potensi] masalah serta
% keterbatasannya. Jika menyangkut penelitian dengan makhluk hidup, maka dibutuhkan adanya \textit{ethical clearance}, di bagian ini hal itu akan dibahas. Demikian juga tentang keterbatasan ataupun masalah yang akan timbul.
